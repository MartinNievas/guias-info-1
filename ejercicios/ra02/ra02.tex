\section{RA2}

\subsection*{Resultados de aprendizaje}
Emplear las herramientas de desarrollo adecuadas para la resolución de problemas a partir de una consigna dada.

\subsection*{Contenido según programa}
Herramientas de desarrollo: El Compilador. El enlazador (Linker). Su relación con el Hardware y el Sistema Operativo. Introducción al desarrollo de software: Programa fuente, programa objeto, programa ejecutable.  Fases de compilación y vinculación de programas en el entorno de un sistema operativo.

\setcounter{subsection}{1}
\subsection{Ejercicios:}

\subsubsection{Responda las siguientes preguntas}
\begin{enumerate}[a)]
  \item Los programas en C normalmente son escritos en un:
  \item ¿Que programa es el encargado de combinar la salida del compilador, con las funciones de bibliotecas para producir el archivo ejecutable?
  \item ¿Cual es el programa que carga en memoria el programa ejecutable, desde el disco?
  \item ¿Como se llama el programa encargado de convertir los programas escritos en lenguajes de alto nivel, al lenguaje 
    de máquina?
  \item ¿Quien es el encargado de colocar un programa en memoria para que pueda ser ejecutado?
  \item ¿Cual es la unidad mínima de información en una computadora?
\end{enumerate}

\subsubsection{Ordenar de forma correcta}
Se desea compilar el programa \texttt{saludo.c} utilizando la línea de comandos (terminal) mediante el programa gcc.
Dada la siguiente lista de pasos, ordene los mismo para producir el archivo ejecutable \texttt{saludo.out}.
Tenga en cuenta que alguno de los comandos pueden ser incorrectos.
Recuerde: \textbf{pre}-procesador, \textbf{com}pilador, \textbf{en}lazador.

\begin{enumerate}
  \item \texttt{gcc -c programa.i}
  \item \texttt{gcc saludo.o -o saludo.c}
  \item \texttt{gcc saludo.c -c saludo}
  \item \texttt{gcc saludo.i -o saludo.out}
  \item \texttt{gcc -E programa.c > programa.i}
  \item \texttt{gcc -c programa.i}
  \item \texttt{gcc -E programa.c > programa.out}
  \item \texttt{gcc programa.o -o saludo.out}
\end{enumerate}

