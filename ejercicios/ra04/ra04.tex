\section{RA4}
\subsection*{Resultados de aprendizaje}
Desarrollar programas en lenguaje C para familiarizarse con los tipos fundamentales de datos y los operadores aritméticos considerando programas simples pero completos.

\subsection*{Contenido según programa}
Introducción al lenguaje C.  Elementos del lenguaje C.  Introducción a la sintaxis del lenguaje C. Primer ejemplo: Hola Mundo. Identificación de los elementos de sintaxis. Uso del compilador. Tipos de datos, tamaño, y declaraciones.  Constantes. Declaraciones.  Operadores aritméticos, relacionales y lógicos. Cast.  Jerarquía de operadores.  Operadores de evaluación (expresiones condicionales).  Operadores de Asignación.  Precedencia.  Preprocesador. Archivos de cabecera. Encabezado stdio.h. Entrada y salida con formato. Funciones básicas de entrada salida: scanf, printf, getch, getchar.

\subsection*{Introducción al lenguaje C.}
\setcounter{subsection}{3}
\subsection{Ejercicios:}

\subsubsection{Completar los espacios en blanco}
\begin{enumerate}
  \item Todo programa en C comienza con la ejecución de la función \underspace.
  \item Todos los cuerpos de las funciones comienzan con \underspace y terminan con \underspace
  \item Todas las declaraciones terminan con \underspace.
  \item La función \underspace de la biblioteca estándar permite mostrar información en la pantalla.
  \item La función \underspace de la biblioteca estándar permite ingresar información desde el teclado.
\end{enumerate}

\subsubsection{Programas}
Escribir un programa en C que resuelva los siguientes problemas (uno por cada enunciado)
\begin{itemize}
  \item Definir las variables, ``unavariable'', ``p345'', ``numero''.
  \item Escribir un mensaje en pantalla, solicitando al usuario ingresar un número entero. Se debe imprimir la
    solicitud, en la cual al final debe incluir dos puntos y dejarse un espacio.
  \item Solicitar al usuario ingresar un número entero y almacenarlo en una variable llamada ``num''.
  \item Imprimir en pantalla el mensaje ``Buen día''.
\end{itemize}

\subsubsection{Ejercicio}
Realizar, para cada enunciado, un programa en C.
\begin{itemize}
  \item Asignar la suma de las variables \textbf{a} y \textbf{b} en \textbf{c}.
  \item Leer 3 números enteros desde el teclado, y almacenarlos en las variables \textbf{p}, \textbf{q} y \textbf{r}.
\end{itemize}

\subsubsection{Verdadero o falso}
Indicar si las siguientes afirmaciones son verdaderas o falsas. Si son falsas, indicar porqué:
\begin{itemize}
  \item La función \textbf{printf} siempre imprime al comienzo de una nueva línea.
  \item Los comentarios son mostrados en la pantalla cuando el programa se está ejecutando.
  \item La secuencia de escape \textbackslash \textbf{n} en la función \textbf{printf}, provoca un salto de línea.
  \item Todas las variables deben definirse antes de ser utilizadas.
  \item El operador de resto (\%) solamente puede ser utilizado en números enteros.
  \item Las variables \textbf{numero} y \textbf{NUmerO} son idénticas.
  \item Los operadores aritméticos \textbf{/,*,+,-,\%} tienen todos el mismo nivel de precedencia.
\end{itemize}

% \subsubsection{Selección}
\subsubsection{Ejercicio 3}
Escribir un algoritmo para calcular la distancia recorrida (m) por un móvil que se desplaza con velocidad constante (m/s) durante un tiempo (s).
La velocidad y el tiempo serán ingresadas por el usuario.

\subsubsection{Ejercicio 4}
Escribir un algoritmo para obtener el promedio simple de un estudiante a partir de las tres notas parciales.
Las notas serán introducidas una a una por el usuario.

\subsubsection{Ejercicio 5}
En un local se hace un descuento del $\mathrm{\%20}$ cuando la compra supera los \$ 1000. Escribir un algoritmo que calcule el precio a pagar por el cliente
teniendo como dato el valor de la compra.

\subsubsection{Ejercicio 6}
Escribir un algoritmo que determine si un número $n$ tiene tres cifras. El usuario debe ingresar el número $n$.

\subsubsection{Ejercicio 7}
Escribir un algoritmo que solicite ingresar dos números $n1$ y $n2$. Si el primero es mayor que el segundo mostrar la suma de ambos, por otro lado si el segundo es mayor al primero, mostrar el producto entre los números. En caso de que sean iguales imprimir ``Los números son iguales''.

