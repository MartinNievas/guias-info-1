\section{Punteros en lenguaje C.}
\setcounter{subsection}{8}

\subsection*{Paso de parámetros por referencia}
Se solicita un programa que cumpla con lo siguiente:
\begin{itemize}[a)]
  \item Requisito 1
    El usuario debe ingresar \textbf{N} valores en un arreglo unidimensional. Los valores ingresados solo pueden ser positivos (incluyendo el cero) menores que 100.

    Se debe realizar la validación en una función que reciba el parámetro por referencia.

  \item Requisito 2
    Una vez cargados los \textbf{N} valores se debe controlar que no haya números primos. Si existiesen deben ser incrementados en una unidad. Implementar esto en una función,
    que debe recibir el valor que debe ser chequeado por referencia.

  \item Requisito 3
    Debe ordenarse el arreglo de mayor a menor utilizando el método de la burbuja. Debe implementarse una función llamada \textit{swap}
    que reciba por referencia los dos elementos del arreglo que deben intercambiarse para lograr el ordenamiento.

  \item Requisito 4
    Debe imprimirse el arreglo antes y después del ordenamiento.


    \hspace{-5mm}El siguiente programa de ejemplo puede servir de ayuda:

    \lstset{inputencoding=utf8/latin1}
    \lstinputlisting[style=customc]{code/ejercicio_demo.c}
\end{itemize}
