\section{Sistemas de numeración y representación numérica. Aritmética binaria.}
\setcounter{subsection}{2}

\subsubsection{Completar los espacios en blanco:}
\begin{tabular}{ccc}
  Decimal&Binario&Hexadecimal\\
  0&00000000&0x00 \\
  158& &\underspace\\
  145&\underspace &\underspace \\
  \underspace&1010 1110&\underspace \\
  \underspace&0011 1100&\underspace \\
  \underspace&1111 0001&\underspace \\
\end{tabular}

\subsubsection{Suma de números (2.4)}
Realizar las siguientes operaciones, considerando números signados. Expresar el resultado en decimal.
\vspace{10mm}

\begin{tabular}{lc}
$0x3D + 0x40 $&=\underspace\\
$0x3D - 0x20 $&=\underspace\\
$01110110 - 00101101$&=\underspace\\
$00011110 + 01101101$&=\underspace\\

\end{tabular}

\subsubsection{Suma de números (2.4)}
Sin convertir a binario, sumar los siguientes números.

\begin{tabular}{lc}
$0x605C + 0x5 $&=\underspace\\
$0x605C - 0x20 $&=\underspace\\
$0x605C + 32 $&=\underspace\\
$0x60fA - 0x605C $&=\underspace\\
\end{tabular}

\subsubsection{Conversiones}
Realizar las siguientes conversiones:
\begin{itemize}
  \item Convierta el binario 110011010010 en octal y en hexadecimal.
  \item Convierta el número 0xFACE hexadecimal a binaria.
  \item Convierte octal 3016 a binario.
  \item Convierta 0xCAFE hexadecimales a octal.
  \item Convertir binario 1001010 a decimal.
  \item Convertir octal 513 a decimal.
  \item Convierta hexadecimal AFD4 a decimal.
  \item Convierta el decimal 177 a binario, a octal y a hexadecimal.
  \item Calcule la representación binaria del decimal 417. Luego el complemento de 417 y el complemento de dos de 417.
  \item ¿Cuál es el resultado cuando un número y su complemento de dos se suman?
\end{itemize}

% \pagebreak
\subsubsection{ Operaciones Booleanas (2.8)}
Completar los espacios, evaluando las operaciones Booleanas:\\

\begin{tabular}{cc}
  Operación&Resultado\\
  a&[01101010]\\
  b&[11000101]\\
  $\sim$ a&\underspace \\
  $\sim$ b&\underspace \\
  a \& b&\underspace\\
  a $\mid$ b&\underspace\\
  a \^{} b&\underspace\\
\end{tabular}

\subsubsection{Operadores binarios (2.14)}
Realizar las siguientes operaciones binarias, teniendo en cuenta los números anteriormente utilizados.\\

\begin{tabular}{cccc}
  Expresión&Valor&Expresión&Valor\\
  a \& b&\underspace &a \&\& b &\underspace\\
  b $\mid$ b&\underspace& a $\mid\mid$ b &\underspace\\
  $\sim$ a $\mid$ $\sim$ b&\underspace& !a $\mid\mid$ !b &\underspace \\
\end{tabular}


\subsubsection{Desplazamiento de bits (2.16)}
Completar con las representaciones correspondientes para cada desplazamiento.\\

\begin{tabular}{cccccccc}
  \multicolumn{2}{c}{ }& &\multicolumn{2}{c}{Lógico}& &\multicolumn{2}{c}{Lógico}\\
  \multicolumn{2}{c}{a}& &\multicolumn{2}{c}{a$<<$2}& &\multicolumn{2}{c}{a$>>$3}\\
  \cline{1-2}\cline{4-5}\cline{7-8}\\
  Hex&Binario&&Hex&Binario&&Hex&Binario\\
  0xD4&\underspace&&\underspace &\underspace &&\underspace &\underspace\\
  0x64&\underspace&&\underspace &\underspace &&\underspace &\underspace\\
  0xD4&\underspace&&\underspace &\underspace &&\underspace &\underspace\\
  0x44&\underspace&&\underspace &\underspace &&\underspace &\underspace\\
\end{tabular}

\subsubsection{Truncado}
En la siguiente tabla escribir las equivalencias de los números suponiendo que sólo se tienen en cuenta 3 bits\\


\begin{tabular}{cccccccc}
  \multicolumn{2}{c}{Hex} & &\multicolumn{2}{c}{No signado}& &\multicolumn{2}{c}{Complemento a 2}\\
  Original&Truncado&&Original&Truncado&&Original&Truncado\\
  \cline{1-2}\cline{4-5}\cline{7-8}\\
  0x1&1          &&1  &\underspace&&1 & \underspace \\
  0x3&3          &&3  &\underspace&&3 & \underspace \\
  0x5&\underspace&&5  &\underspace&&5 & \underspace \\
  0xC&\underspace&&12 &\underspace&&12& -4\\
  0xE&6          &&14 &\underspace&&14& -2\\
\end{tabular}

\subsubsection{Suma}
Realizar las siguientes sumas utilizando números signados de 4 bits:
\begin{itemize}
  \item $ -8-5= $
  \item $-8+5$
  \item $2+5$
  \item $5+5$
\end{itemize}

% \pagebreak
\subsubsection{Interpretación}
Realizar las siguientes sumas e interpretar el resultado como números signados o no signados:\\

\begin{tabular}{ccccc}
x&y&bin&x+y&x+y\\
&&&(4 bits)&(5bits)\\
11000&11000&\underspace&\underspace&\underspace\\
&&&&\\
10111&01000&\underspace&\underspace&\underspace\\
&&&&\\
00010&00101&\underspace&\underspace&\underspace\\
&&&&\\
01100&00100&\underspace&\underspace&\underspace\\
\end{tabular}

