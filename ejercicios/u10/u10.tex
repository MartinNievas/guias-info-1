\section{Manejo de archivos en C.}
\setcounter{subsection}{10}

\subsection*{Preguntas:}
\begin{itemize}[a)]
  \item Completar los espacios en blanco:\\
    \begin{enumerate}
      \item La función \underspace cierra un archivo.
      \item La función \underspace lee los datos de un archivo de manera similar a cómo scanf lee desde stdin.
      \item La función \underspace lee un caracter de un archivo especificado.
      \item La función \underspace reposiciona el puntero de posición del archivo a una ubicación específica en el archivo.
    \end{enumerate}
\end{itemize}
\subsubsection{Ejercicio 1}
Indique cuáles de los siguientes son verdaderos y cuáles son falsos.\\
Si es falso, explique por qué.
\begin{itemize}[a)]
  \item La función fscanf no se puede usar para leer datos de la entrada estándar.
  \item Debe usar explícitamente fopen para abrir la entrada estándar, la salida estándar y las secuencias de error estándar.
  \item Un programa debe llamar explícitamente a la función fclose para cerrar un archivo.
  \item Si el puntero de posición del archivo apunta a una ubicación en un archivo secuencial que no sea el comienzo del archivo, el archivo debe cerrarse y volverse a abrir para leerlo desde el principio del archivo.
  \item La función fprintf puede escribir en la salida estándar.
  \item Los datos en archivos de acceso secuencial siempre se actualizan sin sobrescribir otros datos.
  \item No es necesario buscar a través de todos los registros en un archivo de acceso aleatorio para encontrar un registro específico.
  \item Los registros en archivos de acceso aleatorio no tienen una longitud uniforme.
  \item La función fseek puede buscar solo en relación con el comienzo de un archivo.
\end{itemize}



\subsubsection*{Ejercicio 2} 
Encuentre el error en cada uno de los siguientes segmentos del programa y explique cómo corregirlo.
\begin{enumerate}[a)]
  \item El archivo al que hace referencia fPtr ("pagos.dat") no se ha abierto.
    \lstset{inputencoding=utf8/latin1}
    \lstinputlisting[style=customc]{code/ejercicio2a.c}
  \item \ \ \ 
    \lstinputlisting[style=customc]{code/ejercicio2b.c}
  \item La siguiente declaración debería leer un registro del archivo "pagos.dat". 
    El puntero de archivo payPtr se refiere a este archivo, y el puntero de archivo recPtr se refiere al archivo "recibir.dat"
    \lstinputlisting[style=customc]{code/ejercicio2c.c}
  \item El archivo "tools.dat" debe abrirse para agregar datos al archivo sin descartar los datos actuales.
    \lstinputlisting[style=customc]{code/ejercicio2d.c}
  \item El archivo "course.dat" debe abrirse para agregarlo sin modificar el contenido actual del archivo.
    \lstinputlisting[style=customc]{code/ejercicio2e.c}
\end{enumerate}



