\section{Control de flujo en lenguaje~C.}
\setcounter{subsection}{4}

\subsection{if, if else}

\subsubsection{Solución 0} 

\lstset{inputencoding=utf8/latin1}
\lstinputlisting[style=customc]{code/ejercicio0.c}
{\small
  \lstset{inputencoding=utf8/latin1}
  \lstinputlisting[backgroundcolor = \color{lightgray}]{code/ejercicio0_salida.mn}
}

\subsubsection{Ejercicio 1} 
Realizar un programa que determine el mayor entre dos números ingresados por el usuario
{\small
  \lstset{inputencoding=utf8/latin1}
  \lstinputlisting[backgroundcolor = \color{lightgray}]{code/ejercicio1_salida.mn}
}

\subsubsection{Ejercicio 2} 
Realizar un programa que determine el mayor entre tres números ingresados por el usuario. En caso de que trés o dos números sean iguales y los mayores, es indistinta la elección del mayor.
{\small
  \lstset{inputencoding=utf8/latin1}
  \lstinputlisting[backgroundcolor = \color{lightgray}]{code/ejercicio2_salida.mn}
}

\subsubsection{Ejercicio 3} 
Realizar un programa que controle el encendido de un ventilador en base a la temperatura de encendido y la temperatura ambiente proporcionada por el usuario. Se debe imprimir en pantalla el estado del ventilador luego de ingresar los datos.
{\small
  \lstset{inputencoding=utf8/latin1}
  \lstinputlisting[backgroundcolor = \color{lightgray}]{code/ejercicio3_salida.mn}
}

\subsubsection{Ejercicio 4} 
Realizar un programa que solicite ingresar las notas de dos parciales. Si desaprobó un parcial el programa debe solicitar ingresar la nota del recuperatorio. En base a las notas obtenidas calcular el promedio y determinar la condición académica (Promoción mayor 8, desaprobado menor a 6, lo demás aprobado). Solo se permite recuperar un solo parcial. La nota del recuperatorio se promedia con las notas del parcial y está permitido promocionar si recuperó.
{\small
  \lstset{inputencoding=utf8/latin1}
  \lstinputlisting[backgroundcolor = \color{lightgray}]{code/ejercicio4_salida.mn}
}

\subsubsection{Ejercicio 5} 
Realizar un programa que dada la duracion en minutos de una llamada, permita calcular el costo,considerando:\\
-Hasta tres minutos el costo es 0.50 por minuto
-Por encima de tres minutos es 1.5 fijo más 0.2 por cada minuto adicional a los tres primeros
{\small
  \lstset{inputencoding=utf8/latin1}
  \lstinputlisting[backgroundcolor = \color{lightgray}]{code/ejercicio5_salida.mn}
}



\subsection*{Operadores II, \&\&} 
Realizar un programa que determine si 3 números ingresados son distintos entre ellos.
\subsubsection{Solución 0} 

\lstset{inputencoding=utf8/latin1}
\lstinputlisting[style=customc]{code/ejercicio0.c}
{\small
  \lstset{inputencoding=utf8/latin1}
  % \lstinputlisting[backgroundcolor = \color{lightgray}]{code/ejercicio0_salida.mn}
}

\subsubsection{Ejercicio 1} 
Realizar un programa que determine si un número ingresado por el usuario está en el rango 10-100. En caso de no estar en el rango, el programa de be informarlo y terminar. Si el número está dentro del rango, el programa debe determinar si el número es par.
{\small
  \lstset{inputencoding=utf8/latin1}
  \lstinputlisting[backgroundcolor = \color{lightgray}]{code/ejercicio1_operadores_salida.mn}
}

\subsubsection{Ejercicio 2} 
Escribir un programa que determine el mayor de 3 números ingresados. Si los números son iguales, se debe imprimir un mensaje que lo indique.
{\small
  \lstset{inputencoding=utf8/latin1}
  \lstinputlisting[backgroundcolor = \color{lightgray}]{code/ejercicio2_operadores_salida.mn}
}

\subsubsection{Ejercicio 3} 
Realizar un programa que solicite ingresar el valor nominal de resistencia (en ohms) y la tolerancia (valor entero en porcentaje). Una vez cargados estos valores, el programa debe solicitar al usuario que ingrese un valor de resistencia real y determine si la misma está dentro de los márgenes de tolerancia. 
{\small
  \lstset{inputencoding=utf8/latin1}
  \lstinputlisting[backgroundcolor = \color{lightgray}]{code/ejercicio3_operadores_salida.mn}
}

\subsubsection{Ejercicio 4} 
Realizar un programa que determine si el caracter ingresado es un número o una letra. Ayuda: buscar ``tabla ASCII''.
{\small
  \lstset{inputencoding=utf8/latin1}
  \lstinputlisting[backgroundcolor = \color{lightgray}]{code/ejercicio4_operadores_salida.mn}
}

\subsubsection{Ejercicio 5} 
Modificar el programa anterior para que ahora solicite ingresar el color de las bandas de colores y la medición real de la resisitencia. Con estos valores el programa debe determinar si se encuentra dentro de los valores de tolerancia o no. Considerar el caso de resistencias de 4 bandas de colores, donde las primeras tres indican el valor de resistencia y el cuarto la tolerancia (dorado $ \pm 5 $, plateado $\pm 10$, rojo $\pm 2$ y marrón $\pm 1$. Los colores de las bandas serán ingresados en forma de caracteres. Cada caracter representa un color.
\begin{itemize}
  \item \textbf{N}egro
  \item \textbf{M}arrón
  \item \textbf{R}ojo
  \item naran\textbf{J}a
  \item \textbf{A}marillo
  \item \textbf{V}erde
  \item a\textbf{Z}ul
  \item vio\textbf{L}eta
  \item \textbf{G}ris
  \item \textbf{B}lanco
  \item \textbf{D}orado
  \item \textbf{P}lateado
  \end{itemize}
{\small
  \lstset{inputencoding=utf8/latin1}
  \lstinputlisting[backgroundcolor = \color{lightgray}]{code/ejercicio5_operadores_salida.mn}
}

\subsection*{Estructura repetitiva for}
Todos los ejercicios deben utilizar al menos una estructura \textbf{for}.
Algunos ejercicios pueden requerir utilizar estructuras de condición.

\subsubsection{Ejercicio 0} 
Realizar un programa que imprima los números desde el 5 hasta el 0 y luego vuelva hasta el 5 como en el siguiente ejemplo (se debe utilizar al menos una estructura for)

\subsubsection{Solución 0} 

\lstset{inputencoding=utf8/latin1}
\lstinputlisting[style=customc]{code/ejercicio0.c}
{\small
  \lstset{inputencoding=utf8/latin1}
  \lstinputlisting[backgroundcolor = \color{lightgray}]{code/ejercicio0_salida.mn}
}
\subsubsection{Solución 0 (con un solo for)} 
\lstinputlisting[style=customc]{code/ejercicio0_un.c}
{\small
  \lstset{inputencoding=utf8/latin1}
  \lstinputlisting[backgroundcolor = \color{lightgray}]{code/ejercicio0_salida.mn}
}

\subsubsection{Preguntas}
\begin{itemize}
  \item La iteración controlada por contador también se conoce como iteración \underspace porque se sabe de antemano cuántas veces se ejecutará el bucle.
  \item La iteración controlada por centinela también se conoce como iteración \underspace porque no se sabe de antemano cuántas veces se ejecutará el bucle.
  \item En la iteración controlada por contador, se usa un(una)\underspace  para contar el número de veces que se debe repetir un grupo de instrucciones.
\end{itemize}

\subsubsection{Analizar código}
Encontrar el error en las siguientes códigos:
\begin{enumerate}
  \item .
  \lstinputlisting[style=customc]{code/ejercicio_rep_a.c}
  \item .
  \lstinputlisting[style=customc]{code/ejercicio_rep_b.c}
\end{enumerate}

\subsubsection{Ejercicio 1} 
Modificar el programa anterior para que imprima la progresión de números partiendo de un número $n$ positivo ingresado por el usuario. 
{\small
  \lstset{inputencoding=utf8/latin1}
  \lstinputlisting[backgroundcolor = \color{lightgray}]{code/ejercicio1_rep_salida.mn}
}

\subsubsection{Ejercicio 2} 
Realizar un programa que utilice una entructura \textbf{for} e impmrima la siguiente salida:
{\small
  \lstset{inputencoding=utf8/latin1}
  \lstinputlisting[backgroundcolor = \color{lightgray}]{code/ejercicio2_rep_salida.mn}
}

\subsubsection{Ejercicio 3} 
Realizar un programa que utilice una entructura \textbf{for} e impmrima la siguiente tabla:
{\small
  \lstset{inputencoding=utf8/latin1}
  \lstinputlisting[backgroundcolor = \color{lightgray}]{code/ejercicio3_rep_salida.mn}
}

\subsubsection{Ejercicio 4} 
Realizar un programa que utilice dos entructuras \textbf{for} para imprimir una matriz, donde la cantidad de filas y columnas son ingresados por el usuario como la siguiente:
{\small
  \lstset{inputencoding=utf8/latin1}
  \lstinputlisting[backgroundcolor = \color{lightgray}]{code/ejercicio4_rep_salida.mn}
}

\subsubsection{Ejercicio 5} 
Escribir un programa que imprima todos los números enteros pares entre el 0 y $n$, donde $n$ es un número entero ingresado por el usuario.
{\small
  \lstset{inputencoding=utf8/latin1}
  \lstinputlisting[backgroundcolor = \color{lightgray}]{code/ejercicio5_rep_salida.mn}
}

\subsubsection{Ejercicio 6} 
Escribir un programa que determine el mayor de 10 números enteros ingresados por el usuario.
{\small
  \lstset{inputencoding=utf8/latin1}
  \lstinputlisting[backgroundcolor = \color{lightgray}]{code/ejercicio6_rep_salida.mn}
}

\subsubsection{Ejercicio 7} 
Modificar el programa del ejercicio anterior para que determine también el mínimo número ingresado.
{\small
  \lstset{inputencoding=utf8/latin1}
  \lstinputlisting[backgroundcolor = \color{lightgray}]{code/ejercicio7_rep_salida.mn}
}

\subsubsection{Ejercicio 8} 
Realizar un programa que calcule la tabla de multiplicar de un número $n$ ingresado por el usuario.
{\small
  \lstset{inputencoding=utf8/latin1}
  \lstinputlisting[backgroundcolor = \color{lightgray}]{code/ejercicio8_rep_salida.mn}
}

\subsubsection{Ejercicio 9} 
Realizar un programa que calcule el factorial de un número $n$ ingresado por el usuario. Donde $0<n<10$.
{\small
  \lstset{inputencoding=utf8/latin1}
  \lstinputlisting[backgroundcolor = \color{lightgray}]{code/ejercicio9_rep_salida.mn}
}

\subsubsection{Ejercicio 10} 
Realizar un programa que calcule la potencia $m$ de número $n$, donde $m$ y $n$ son ingresado por el usuario. Operación: $n^m$.
{\small
  \lstset{inputencoding=utf8/latin1}
  \lstinputlisting[backgroundcolor = \color{lightgray}]{code/ejercicio10_rep_salida.mn}
}

\subsection*{Estructuras repetitivas while y do..while}
Todos los ejercicios deben utilizar al menos una estructura \textbf{for}.
Repetir ahora utilizando estructura \textbf{while} y \textbf{do..while}.

