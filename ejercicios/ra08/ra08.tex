\section{RA8}
\subsection*{Resultados de aprendizaje}
Utilizar punteros para uso general y en particular para pasar argumentos teniendo en cuenta que es una de las características más poderosa del lenguaje C.

\subsection*{Contenido según programa}
Punteros en lenguaje C.  Concepto de puntero.  Concepto de dirección.  Operadores unarios.  Aritmética de punteros.  Relación entre punteros y arreglos. Inicialización de punteros. Implementación de llamadas a función por referencia. Punteros a puntero. Arreglo de punteros. Ordenamiento de estructuras utilizando arreglo de punteros.  Inicialización de punteros y reserva de espacio en memoria: malloc y free.  Punteros vs. Arreglos multidimensionales.  Argumentos por línea de comandos. Punteros a función.

%--------------------------------------------------------------------------------
%--------------------------------------------------------------------------------
\setcounter{subsection}{8}
\subsection*{Punteros en lenguaje C.}

\subsection*{Paso de parámetros por referencia}

\subsection{Ejercicio}
Se solicita un programa que cumpla con lo siguiente:
\begin{itemize}[a)]
  \item Requisito 1
    El usuario debe ingresar \textbf{N} valores en un arreglo unidimensional. Los valores ingresados solo pueden ser positivos (incluyendo el cero) menores que 100.

    Se debe realizar la validación en una función que reciba el parámetro por referencia.

  \item Requisito 2
    Una vez cargados los \textbf{N} valores se debe controlar que no haya números primos. Si existiesen deben ser incrementados en una unidad. Implementar esto en una función,
    que debe recibir el valor que debe ser chequeado por referencia.

  \item Requisito 3
    Debe ordenarse el arreglo de mayor a menor utilizando el método de la burbuja. Debe implementarse una función llamada \textit{swap}
    que reciba por referencia los dos elementos del arreglo que deben intercambiarse para lograr el ordenamiento.

  \item Requisito 4
    Debe imprimirse el arreglo antes y después del ordenamiento.


    El siguiente programa de ejemplo puede servir de ayuda:
    \pagebreak

    \lstset{inputencoding=utf8/latin1}
    \lstinputlisting[style=customc]{code/ejercicio_demo.c}
\end{itemize}

\subsection{Ejercicio}
Se solicita un programa que cumpla con lo siguiente:
\begin{itemize}[a)]
  \item Requisito 1
    Inicializar un arreglo unidimensional de tamaño\textbf{N} con valores pseudoaleatorios entre el 0 y el 30000

  \item Requisito 2
    Ordenar el arreglo mediante la función \texttt{ordena\_burbuja} la cual recibe una variable entera llamada \texttt{dir} la cual indica el modo de ordenamiento: ``0'' de mayor a menor y ``1'' de menor a mayor.

  \item Requisito 3
    Debe ordenarse el arreglo utilizando el método de la burbuja. Debe implementarse una función llamada \textit{swap} que reciba por referencia los dos elementos del arreglo que deben intercambiarse para lograr el ordenamiento.

  \item Requisito 4
    Debe imprimirse el arreglo antes y después del ordenamiento. Llamar a la función ordenando al arreglo de menor a mayor, y de menor a mayor.

    El siguiente programa de ejemplo puede servir de ayuda:
    \pagebreak

    \lstset{inputencoding=utf8/latin1}
    \lstinputlisting[style=customc]{code/ejercicio_demo2.c}
\end{itemize}
