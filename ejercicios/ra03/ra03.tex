\section{RA3}
\subsection*{Resultados de aprendizaje}
Reconocer las ventajas tecnológicas de los sistemas de numeración en bases no decimales, particularmente la binaria, para el procesamiento en la computadora teniendo en cuenta la ``naturaleza binaria'' del hardware.

\subsection*{Contenido según programa}
Sistemas de numeración y representación numérica.  Aritmética binaria.  Introducción. Los sistemas de numeración y su evolución histórica.  Sistemas de numeración decimal, binario, octal y hexadecimal. Pasajes entre sistemas de números enteros y positivos.  Representación de Números signados: Convenio de signo y magnitud. Convenio de complemento a uno.  Convenio de complemento a dos. Operaciones de adición y de sustracción utilizando el convenio de complemento a dos.  Representación de números fraccionales. Notación punto fijo y punto flotante.  Precisión y truncado.  Errores en notación de punto flotante.  Representación según formato IEEE 754.  Representación de caracteres: Binario Codificado Decimal (BCD), ASCII, Unicode.

\subsection*{Sistemas de numeración y representación numérica. Aritmética binaria.}
\setcounter{subsection}{2}
\subsection{Ejercicios:}

\subsubsection{Completar los espacios en blanco:}
\begin{tabular}{ccc}
  Decimal&Binario&Hexadecimal\\
  0&00000000&$\mathrm{0x00}$ \\
  158& &\underspace\\
  145&\underspace &\underspace \\
  \underspace&1010 1110&\underspace \\
  \underspace&0011 1100&\underspace \\
  \underspace&1111 0001&\underspace \\
  255& &\underspace\\
  187&\underspace &\underspace \\
  \underspace&1101 1010&\underspace \\
  \underspace&0110 1011&\underspace \\
  \underspace&1001 1110&\underspace \\
  1023& &\underspace\\
  421&\underspace &\underspace \\
  \underspace&1011 0010&\underspace \\
  \underspace&1000 1111&\underspace \\
  \underspace&0111 1011&\underspace \\
\end{tabular}

\subsubsection{Suma de números}
Realizar las siguientes operaciones, considerando números signados de 8 bits. Expresar el resultado en decimal.
\vspace{10mm}

\begin{tabular}{lc}
  $\mathrm{0x3D} + \mathrm{0x40} $&=\underspace\\
  $\mathrm{0x3D} - \mathrm{0x20} $&=\underspace\\
  $\mathrm{0x4A} - \mathrm{0x03} $&=\underspace\\
  $\mathrm{0x44} - \mathrm{0x10} $&=\underspace\\
  $\mathrm{01110110} - \mathrm{00101101}$&=\underspace\\
  $\mathrm{00011110} + \mathrm{01101101}$&=\underspace\\
  $\mathrm{01011110} - \mathrm{01111111}$&=\underspace\\
  $\mathrm{01111110} + \mathrm{01101000}$&=\underspace\\
  $\mathrm{0x32} + \mathrm{0x50}$&=\underspace\\
  $\mathrm{0x90} - \mathrm{0x20}$&=\underspace\\
  $\mathrm{0x78} + \mathrm{0xA3}$&=\underspace\\
  $\mathrm{0xA2} - \mathrm{0x5E}$&=\underspace\\
  $\mathrm{11010010} + \mathrm{10100101}$&=\underspace\\
  $\mathrm{11101100} - \mathrm{10100011}$&=\underspace\\
  $\mathrm{01010110} + \mathrm{01110101}$&=\underspace\\
  $\mathrm{01111001} - \mathrm{00110011}$&=\underspace\\

\end{tabular}


\subsubsection{Suma de números (2.6)}
Realizar las siguientes operaciones, considerando números signados de 16 bits. Expresar el resultado en decimal.

\vspace{10mm}
\begin{tabular}{lc}
  $\mathrm{0x1234} + \mathrm{0x5678}$&=\underspace\\
  $\mathrm{0x9ABC} - \mathrm{0x2000}$&=\underspace\\
  $\mathrm{0x7FED} + \mathrm{0xA3CF}$&=\underspace\\
  $\mathrm{0xC2A1} - \mathrm{0x5F12}$&=\underspace\\
  $\mathrm{10101010 11010011} + \mathrm{11001101 11100000}$&=\underspace\\
  $\mathrm{01101011 00110011} - \mathrm{10110101 11011010}$&=\underspace\\
\end{tabular}

\subsubsection{Suma de números}
Sin convertir a binario, sumar los siguientes números.

\begin{tabular}{lc}
  $\mathrm{0x605C} + \mathrm{0x5} $&=\underspace\\
  $\mathrm{0x605C -} + \mathrm{0x20} $&=\underspace\\
  $\mathrm{0x605C +} + \mathrm{32} $&=\underspace\\
  $\mathrm{0x60fA -} + \mathrm{0x605C} $&=\underspace\\
  $\mathrm{0x43} - \mathrm{0x12}$&=\underspace\\
  $\mathrm{0xF2} + \mathrm{0x5A}$&=\underspace\\
  $\mathrm{0xE1} - \mathrm{0x93}$&=\underspace\\
  $\mathrm{10101011} + \mathrm{11010010}$&=\underspace\\
  $\mathrm{00011101} - \mathrm{00101100}$&=\underspace\\
\end{tabular}

\subsubsection{Conversiones}
Realizar las siguientes conversiones:
\begin{itemize}
  \item Convierta el binario 110011010010 en octal y en hexadecimal.
  \item Convierta el número $\mathrm{0xFACE}$ hexadecimal a binaria.
  \item Convierte octal $\mathrm{3016}$ a binario.
  \item Convierta 0xCAFE hexadecimales a octal.
  \item Convertir binario $\mathrm{1001010}$ a decimal.
  \item Convertir octal $\mathrm{513}$ a decimal.
  \item Convierta hexadecimal $\mathrm{0xAFD4}$ a decimal.
  \item Convierta el decimal $\mathrm{177}$ a binario, a octal y a hexadecimal.
  \item Calcule la representación binaria del decimal $\mathrm{417}$. Luego el complemento de $\mathrm{417}$ y el complemento de dos de $\mathrm{417}$.
  \item ¿Cuál es el resultado cuando un número y su complemento de dos se suman?
  \item Convierta el número $\mathrm{0x1234}$ hexadecimal a decimal.
  \item Convertir binario $\mathrm{10101010}$ a octal.
  \item Convierta el decimal $\mathrm{543}$ a hexadecimal y a binario.
  \item Calcule la representación en base 16 del número $\mathrm{2000}_{10}$.
  \item Convierta el número $\mathrm{0x12345678}$ hexadecimal a binario.
  \item Convertir octal $\mathrm{7777}$ a decimal.
  \item Calcule la representación en base 10 del número $\mathrm{100101011010}_{2}$.
  \item Convierte el número $\mathrm{0xFFFE0000}$ hexadecimal a decimal.
\end{itemize}

\subsubsection{Conversión decimal binario}
Convertir a binario los siguientes números decimales. Utilizar 10 bits en total.
\begin{itemize}
  \item $13.44$
  \item $23.25$
  \item $51.1$
  \item $4.34$
  \item $9.675$
  \item $17.14$
  \item $15.128$
\end{itemize}

\subsubsection{Conversión binario decimal fraccionario}
Convertir a decimal los siguientes números binarios.
\begin{itemize}
  \item $\mathrm{1110.10}$
  \item $\mathrm{1010.1011}$
  \item $\mathrm{110010.111}$
  \item $\mathrm{1000010.011}$
  \item $\mathrm{101010.101}$
  \item $\mathrm{101111.111}$
  \item $\mathrm{101011.0001}$
\end{itemize}

\subsubsection{Truncado}
En la siguiente tabla escribir las equivalencias de los números suponiendo que sólo se tienen en cuenta 3 bits\\

\begin{tabular}{cccccccc}
  \multicolumn{2}{c}{Hex} & &\multicolumn{2}{c}{No signado}& &\multicolumn{2}{c}{Complemento a 2}\\
  Original&Truncado&&Original&Truncado&&Original&Truncado\\
  \cline{1-2}\cline{4-5}\cline{7-8}\\
  $\mathrm{0x1}$&1          &&1  &\underspace&&1 & \underspace \\
  $\mathrm{0x3}$&3          &&3  &\underspace&&3 & \underspace \\
  $\mathrm{0x5}$&\underspace&&5  &\underspace&&5 & \underspace \\
  $\mathrm{0xC}$&\underspace&&12 &\underspace&&12& -4\\
  $\mathrm{0xE}$&6          &&14 &\underspace&&14& -2\\
\end{tabular}

\subsubsection{Suma}
Realizar las siguientes sumas utilizando números signados de 4 bits:
\begin{itemize}
  \item $ -8-5= $
  \item $-8+5 = $
  \item $2+5  = $
  \item $5+5  = $
\end{itemize}

\subsubsection{Interpretación}
Realizar las siguientes sumas e interpretar el resultado como números signados o no signados:\\

\begin{tabular}{ccccc}
x&y&bin&x+y&x+y\\
&&&(4 bits)&(5bits)\\
11000&11000&\underspace&\underspace&\underspace\\
&&&&\\
10111&01000&\underspace&\underspace&\underspace\\
&&&&\\
00010&00101&\underspace&\underspace&\underspace\\
&&&&\\
01100&00100&\underspace&\underspace&\underspace\\
&&&&\\
01110&01100&\underspace&\underspace&\underspace\\
&&&&\\
01101&00110&\underspace&\underspace&\underspace\\
\end{tabular}

\subsubsection{Convertir de decimal a IEEE754}
Convertir los siguientes números en representación decimal a float IEEE754
\begin{enumerate}[a)]
  \item $1.25$
  \item $7.125$
  \item $127.575$
  \item $-1.25$
  \item $-1.675$
  \item $-0.25$
  \item $-0.01$
  \item $63.125$
  \item $-0.0123$
\end{enumerate}

\subsubsection{Convertir desde IEEE754 a decimal}
\begin{enumerate}[a)]
  \item 01000001110000010000000000000000
  \item 01000010000000010000000000000000
  \item 11000001010001100000000000000000
  \item 11000010101110010000000000000000
  \item 01000010100111000100000000000000
  \item 01000010100111001100000000000000
\end{enumerate}

% \pagebreak
\subsubsection{ Operaciones Booleanas (2.8)}
Completar los espacios, evaluando las operaciones Booleanas:\\

\begin{tabular}{cc}
  Operación&Resultado\\
  a&[01101010]\\
  b&[11000101]\\
  $\sim$ a&\underspace \\
  $\sim$ b&\underspace \\
  a \& b&\underspace\\
  a $\mid$ b&\underspace\\
  a \^{} b&\underspace\\
\end{tabular}

\subsubsection{Operadores binarios (2.14)}
Realizar las siguientes operaciones binarias, teniendo en cuenta los números anteriormente utilizados.\\

\begin{tabular}{cccc}
  Expresión&Valor&Expresión&Valor\\
  a \& b&\underspace &a \&\& b &\underspace\\
  b $\mid$ b&\underspace& a $\mid\mid$ b &\underspace\\
  $\sim$ a $\mid$ $\sim$ b&\underspace& !a $\mid\mid$ !b &\underspace \\
\end{tabular}


\subsubsection{Desplazamiento de bits (2.16)}
Completar con las representaciones correspondientes para cada desplazamiento.\\

\begin{tabular}{cccccccc}
  \multicolumn{2}{c}{ }& &\multicolumn{2}{c}{Lógico}& &\multicolumn{2}{c}{Lógico}\\
  \multicolumn{2}{c}{a}& &\multicolumn{2}{c}{a$<<$2}& &\multicolumn{2}{c}{a$>>$3}\\
  \cline{1-2}\cline{4-5}\cline{7-8}\\
  Hex&Binario&&Hex&Binario&&Hex&Binario\\
  $\mathrm{0xD4}$&\underspace&&\underspace &\underspace &&\underspace &\underspace\\
  $\mathrm{0x64}$&\underspace&&\underspace &\underspace &&\underspace &\underspace\\
  $\mathrm{0xD4}$&\underspace&&\underspace &\underspace &&\underspace &\underspace\\
  $\mathrm{0x44}$&\underspace&&\underspace &\underspace &&\underspace &\underspace\\
\end{tabular}

